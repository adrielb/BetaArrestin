We used the two member MAPK scaffold model (cite PNAS paper)
as a basis for our model with the following alterations.
First, RAF (MAPKKK) dynamics were removed and 
a basal level of RAF activation was assumed.
Second, we spacially partition the cell into "front" and "back" compartments,
each with subcellular regions of cytoplasm, membrane and vesicles. 
Scaffold is capable of recycling through those three subcellular regions and 
is initially only cytoplasmic. 
MAPK components are cytoplasmic
however they can be recruited to vesicle scaffolds,
which can further catalyze the creation of dually phosphorylated MAPK.

Translocation of cytoplasmic scaffold to the membrane occurs at a rate
proportional to input dose.
Translocation of membrane scaffold to vesicles occurs at a biphasic rate
dependent on membrane scaffold.
The readout for the model is the amount of dually phosphorylated MAPK.

Simulation was carried out through temporal integration of the system to steady
state until an equilibrium tolerance was reached. 

X-p phosphorylation, X-Ph phosphatase, X-k, kinase

