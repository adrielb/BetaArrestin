% Preamble {{{
\documentclass{beamer}
\graphicspath{{../python/}}
\mode<presentation>
{
  \usetheme{Boadilla}
  %\usetheme{Warsaw}
  \setbeamercovered{transparent}
  \setbeamertemplate{navigation symbols}{}
  \setbeamertemplate{background canvas}[vertical shading][bottom=brown!10,top=yellow!10]
  \setbeamertemplate{footline}{}
  \setbeamertemplate{frametitle}[default][center]
  \setbeamerfont{frametitle}{size=\LARGE,series=\bfseries}
  %\setsansfont[BoldFont={Droid Sans Bold}]{Droid Sans}
  %\setbeamerfont{frametitle}{
}
\usepackage[english]{babel}
\usepackage[latin1]{inputenc}
\usepackage{times}
\usepackage[T1]{fontenc}

\newcommand{\plotslide}[2]{
  \begin{frame}
    \frametitle{#1}
    \centering
    \includegraphics[height=\textheight,width=0.9\textwidth,keepaspectratio=true]{#2}
  \end{frame}
}
%}}}

\begin{document}

\plotslide{Model Schematic}{../tikz/flow.pdf}

\iffalse
\plotslide{Constant p4}{constant_p4_func.pdf}
\plotslide{Constant p4}{p4a_sign_change_Sves.pdf}
\plotslide{Constant p4}{p4a_sign_change_MI.pdf}

\begin{frame}
  \frametitle{Constant p4}
  \begin{itemize}
    \item Positive amp is good for native
    \item Negative amp is good for overexpressed
    \item Smoothly change amp w.r.t expression level
  \end{itemize}
\end{frame}

\plotslide{Coupled p4}{coupled_p4_func.pdf}
\plotslide{Coupled p4}{Sves_coupled_p4.pdf}
\plotslide{Coupled p4}{MI_coupled_p4.pdf}

\fi


\plotslide{}{StotVsMAPKpp.pdf}

\plotslide{}{SmemVsp3.pdf}

\begin{frame}
  \frametitle{p4 as a function of Sves and dSves}
  \begin{itemize}
    \itemsep3em
    \item Previously I've demonstrated why p4 had to be a function of dSves
    \item dSves serves as the polarity compass btw front and back of the cell
    \item Now I'm demonstrating why p4 needs to be a function of Sves as
      well
    \item Sves serves as sigmoidal strength factor
      \begin{itemize}
          \item Low Sves: short transition range
          \item High Sves: long transition range
      \end{itemize}
  \end{itemize}
\end{frame}
\plotslide{}{fig-p4func.pdf}
\plotslide{}{fig-p4func3D_0.pdf}
\plotslide{}{fig-p4func3D_1.pdf}

\begin{frame}
  \frametitle{Comparison with experiments}
\end{frame}

\plotslide{MAPKpp Dose Response}{../hao/dose_response_ERK_t.png}
\plotslide{MAPKpp Dose Response}{InputVsMAPKpp.pdf} 

\plotslide{MI Dose Response}{../hao/dose_response_MI_t.png}
\plotslide{MI Dose Response}{InputVsMI.pdf}

\plotslide{Scaffold Response}{../hao/exp-MIvsS_t.png}
\plotslide{Scaffold Response}{ScaffoldVsMAPKppMI.pdf}

\plotslide{Gradient Response}{../hao/gradient_response_t.png}
\plotslide{Gradient Response}{GradientVsMI.pdf}

\plotslide{Nocodazole treatment}{../hao/noco_t.png}
\plotslide{$S_{ves} \rightarrow S_{cyto}, 20\%  p_4 $}{noco.pdf}

\plotslide{Rab11-DN}{../hao/Rab11DN_t.png}
\plotslide{Rab11-DN}{rab11dn.pdf}

%\plotslide{}{../hao/drugs_t.png}

\end{document}


