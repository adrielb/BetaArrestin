\documentclass{beamer}
\graphicspath{{../python/}}
\mode<presentation>
{
  \usetheme{Boadilla}
  %\usetheme{Warsaw}
  \setbeamercovered{transparent}
  \setbeamertemplate{navigation symbols}{}
  \setbeamertemplate{background canvas}[vertical shading][bottom=brown!10,top=yellow!10]
  \setbeamertemplate{footline}{}
  \setbeamertemplate{frametitle}[default][center]
  \setbeamerfont{frametitle}{size=\LARGE,series=\bfseries}
  %\setsansfont[BoldFont={Droid Sans Bold}]{Droid Sans}
  %\setbeamerfont{frametitle}{
}
\usepackage[english]{babel}
\usepackage[latin1]{inputenc}
\usepackage{times}
\usepackage[T1]{fontenc}

\newcommand{\plotslide}[2]{
  \begin{frame}
    \frametitle{#1}
    \centering
    \includegraphics[height=\textheight,width=0.9\textwidth,keepaspectratio=true]{#2}
  \end{frame}
}
\begin{document}

\plotslide{Model Schematic}{../tikz/flow.pdf}


\begin{frame}
  \frametitle{Sves inhibits Rab11 localization}
  \begin{itemize}
    \item p4(dSves) aka Rab11 needs to be a function of the difference between
      vescile scaffold in the front from the back
    \item $dSves = Sves_{front} - Sves_{back}$
    \item in other words, vescile scaffold and Rab11 are mutually inhibitory
    \item Since native dSves is small, need fast p4 switching to amplify
      native MI
    \item Need slow p4 switching for the OE case in order to prevent positive
      feedback locking, otherwise you wont see the chemorepulsion to attraction
      transition
  \end{itemize}
\end{frame}

\plotslide{Fast p4 switching}{n_coupled_p4_func.pdf}
\plotslide{Fast p4 switching}{n_Sves_coupled_p4.pdf}
\plotslide{Fast p4 switching}{n_dSves_coupled_p4.pdf}
\plotslide{Fast p4 switching}{n_MI_coupled_p4.pdf}
\begin{frame}
  \frametitle{Fast p4 switching}
  \begin{itemize}
    \item we get a boost in native MI but no positive MI at high dose
      overexpressed cells
    \item but if we reduce the sigmoidal transition w.r.t. dSves, we can
      amplify at OE levels at the expense of the native case
  \end{itemize}
\end{frame}
\plotslide{Slow p4 switching}{oe_coupled_p4_func.pdf}
\plotslide{Slow p4 switching}{oe_Sves_coupled_p4.pdf}
\plotslide{Slow p4 switching}{oe_dSves_coupled_p4.pdf}
\plotslide{Slow p4 switching}{oe_MI_coupled_p4.pdf}

\begin{frame}
  \frametitle{Sves inhibits Rab11 localization}
  \begin{itemize}
    \item I feel like we can almost get both cases to work
    \item We have a single model that can amplify in the native case and in
      the OE case, just with just a signle parameter change
    \item Another way to interpret this is that vesicle scaffold inhibits
      Rab11 translocation AND reduces its kinetics 
  \end{itemize}
\end{frame}

\iffalse
\plotslide{Constant p4}{constant_p4_func.pdf}
\plotslide{Constant p4}{p4a_sign_change_Sves.pdf}
\plotslide{Constant p4}{p4a_sign_change_MI.pdf}

\begin{frame}
  \frametitle{Constant p4}
  \begin{itemize}
    \item Positive amp is good for native
    \item Negative amp is good for overexpressed
    \item Smoothly change amp w.r.t expression level
  \end{itemize}
\end{frame}

\plotslide{Coupled p4}{coupled_p4_func.pdf}
\plotslide{Coupled p4}{Sves_coupled_p4.pdf}
\plotslide{Coupled p4}{MI_coupled_p4.pdf}


\begin{frame}
  \frametitle{Comparison with experiments}
\end{frame}

\plotslide{}{StotVsMAPKpp.pdf}

\plotslide{}{SmemVsp3.pdf}

\plotslide{}{../hao/dose_response_ERK_t.png}
\plotslide{}{InputVsMAPKpp.pdf}

\plotslide{}{../hao/dose_response_MI_t.png}
\plotslide{}{InputVsMI.pdf}

\plotslide{}{../hao/exp-MIvsS_t.png}
\plotslide{}{ScaffoldVsMAPKppMI.pdf}

\plotslide{}{../hao/gradient_response_t.png}
\plotslide{}{GradientVsMI.pdf}

\plotslide{Nocodazole treatment}{../hao/noco_t.png}
\plotslide{$S_{ves} \rightarrow S_{cyto}, 20\%  p_4 $}{noco.pdf}

\plotslide{Rab11-DN}{../hao/Rab11DN_t.png}
\plotslide{Rab11-DN}{rab11dn.pdf}

%\plotslide{}{../hao/drugs_t.png}
\fi

\end{document}


