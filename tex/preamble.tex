% Preamble
\documentclass{beamer}
\graphicspath{{../python/}}
\mode<presentation>
{
  \usetheme{Boadilla}
  %\usetheme{Warsaw}
  \setbeamercovered{transparent}
  \setbeamertemplate{navigation symbols}{}
  \setbeamertemplate{background canvas}[vertical shading][bottom=brown!10,top=yellow!10]
  \setbeamertemplate{footline}{}
  \setbeamertemplate{items}[circle]
  \setbeamertemplate{frametitle}[default][center]
  \setbeamerfont{frametitle}{size=\LARGE,series=\bfseries}
  %\setsansfont[BoldFont={Droid Sans Bold}]{Droid Sans}
  %\setbeamerfont{frametitle}{
}
\usepackage[english]{babel}
\usepackage[latin1]{inputenc}
\usepackage{times}
\usepackage[T1]{fontenc}

\newcommand{\plotslide}[2]{
  \begin{frame}
    \frametitle{#1}
    \centering
    \includegraphics[height=\textheight,width=0.9\textwidth,keepaspectratio=true]{#2}
  \end{frame}
}

% Adding footnotes 
% http://www.math.umbc.edu/~rouben/beamer/quickstart-Z-H-17.html
% The reference environment takes two arguments, These specify the footnote’s
% position relative to the slide’s top left corner. A Beamer slide is
% 128mm×96mm.
\usepackage[absolute,overlay]{textpos} 
\newenvironment{reference}[2]{% 
  \begin{textblock*}{\textwidth}(#1,#2) 
      \footnotesize\it\bgroup\color{red!50!black}}{\egroup\end{textblock*}} 
