
\plotslide{}{SmemVsp3.pdf}

\begin{frame}{Why is p3 a bell shaped response?}
  \begin{reference}{4mm}{85mm}
    this is a test \\
    another line
  \end{reference}
  \begin{itemize}
    \itemsep3em
    \item Each part of the bell shaped curved is simulated to justify its use
    \begin{itemize}
      \item Constant term
      \item Linear rising term
      \item Just the falling side
      \item Bell shape
      \begin{itemize}
        \item by combining the rising and falling terms 
      \end{itemize}
    \end{itemize}
  \end{itemize}
\end{frame}

\plotslide{}{p3effects-params.pdf}

\plotslide{}{p3effects_Constant.pdf}
\begin{frame}
  \frametitle{Results for constant p3}
  \begin{itemize}
    \itemsep3em
    \item At native levels, we get a saturating increase in MAPKpp, producing
      a negatively sloping MI curve
      \begin{itemize}
          \item Experimentally MI is constant w.r.t. dose
      \end{itemize}
    \item At overexpressed levels, the simulation only produces negative MI
      values
      \begin{itemize}
          \item Experimentally MI goes from negative to positive
      \end{itemize}
  \end{itemize}
\end{frame}

\plotslide{}{p3effects_Rising.pdf}
\begin{frame}
  \frametitle{Results for rising p3}
  \begin{itemize}
    \itemsep3em
    \item At native levels, MAPKpp now rises linearly with dose, producing
      a flat MI response since it is the derivative
      \begin{itemize}
          \item This is what is observed experimentally
      \end{itemize}
    \item At overexpressed levels, the simulation still only produces negative MI
      values
      \begin{itemize}
          \item Experimentally MI goes from negative to positive
      \end{itemize}
  \end{itemize}
\end{frame}

\plotslide{}{p3effects_Falling.pdf}
\begin{frame}
  \frametitle{Results for falling p3}
  \begin{itemize}
    \itemsep3em
    \item At native levels, MAPKpp saturates rapidly just like the constant
      case, producing a decreasing MI response 
      \begin{itemize}
          \item MI should be constant
      \end{itemize}
    \item At overexpressed levels, MAPKpp is now able to rise at high inputs after
      falling, producing a MI response that goes from negative to positive
      \begin{itemize}
          \item Experimentally MI goes from negative to positive
      \end{itemize}
  \end{itemize}
\end{frame}

\plotslide{}{p3effects_Bell.pdf}
\begin{frame}
  \frametitle{Results for bell shaped p3}
  \begin{itemize}
    \itemsep3em
    \item Now the experimental results can be satisfied at both expression
      levels
    \item At native levels, MAPKpp rises linearly, producing constant MI 
    \item At overexpressed levels, MAPKpp has an inverted bell response,  
      producing a MI curve that changes sign
  \end{itemize}
\end{frame}
